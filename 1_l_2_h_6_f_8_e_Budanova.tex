\documentclass[12pt]{article}
\usepackage{amsmath}
\usepackage{geometry}
\usepackage[utf8]{inputenc}
\usepackage[T2A]{fontenc}
\usepackage[russian]{babel}

\geometry{a4paper, margin=1in}

\title{Коллоквиум 1: работа над ошибками}
\author{}
\date{}

\begin{document}

\maketitle

\section*{Задание 1l}
Найти сумму $A + B$ и разность $A - B$ для матриц $A$ и $B$:

\[
A = \begin{pmatrix}
43 & -62 & 76 \\
54 & -10 & 57 \\
90 & -9 & 7
\end{pmatrix}, \quad 
B = \begin{pmatrix}
31 & -7 & -2 \\
-6 & 83 & 2 \\
-28 & 49 & 99
\end{pmatrix}
\]

\[
A + B = 
\begin{pmatrix}
43 + 31 & -62 - 7 & 76 - 2 \\
54 - 6 & -10 + 83 & 57 + 2 \\
90 - 28 & -9 + 49 & 7 + 99
\end{pmatrix} = 
\begin{pmatrix}
74 & -69 & 74 \\
48 & 73 & 59 \\
62 & 40 & 106
\end{pmatrix}
\]

\[
A - B = 
\begin{pmatrix}
43 - 31 & -62 + 7 & 76 + 2 \\
54 + 6 & -10 - 83 & 57 - 2 \\
90 + 28 & -9 - 49 & 7 - 99
\end{pmatrix} = 
\begin{pmatrix}
12 & -55 & 78 \\
60 & -93 & 55 \\
118 & -58 & -92
\end{pmatrix}
\]

\section*{Задание 2h}
Найти произведение $A \cdot B$ и $B \cdot A$ для матриц $A$ и $B$:

\[
A = \begin{pmatrix}
1 & 5 \\
-1 & 2 \\
0 & -3
\end{pmatrix}, \quad 
B = \begin{pmatrix}
-1 & -4 & -1 \\
1 & -2 & 0
\end{pmatrix}
\]

\[
A \cdot B = 
\begin{pmatrix}
1 & 5 \\
-1 & 2 \\
0 & -3
\end{pmatrix} \cdot
\begin{pmatrix}
-1 & -4 & -1 \\
1 & -2 & 0
\end{pmatrix} = 
\begin{pmatrix}
-1 + 5 & -4 - 10 & -1 + 0 \\
1 + 2 & 4 - 4 & 1 + 0 \\
0 - 3 & 0 + 6 & 0 + 0
\end{pmatrix} = 
\begin{pmatrix}
4 & -14 & -1 \\
3 & 0 & 1 \\
-3 & 6 & 0
\end{pmatrix}
\]

\[
B \cdot A = 
\begin{pmatrix}
-1 & -4 & -1 \\
1 & -2 & 0
\end{pmatrix} \cdot
\begin{pmatrix}
1 & 5 \\
-1 & 2 \\
0 & -3
\end{pmatrix} = 
\begin{pmatrix}
-1 + 4 + 0 & -5 - 8 + 3 \\
1 + 2 + 0 & 5 - 4 + 0
\end{pmatrix} = 
\begin{pmatrix}
3 & -10 \\
3 & 1
\end{pmatrix}
\]

\section*{Задание 6f}
Найти скалярное произведение векторов:

\[
\vec{v} = (-2; 10), \quad X[1; -2], \quad Z[7; -7]
\]

\[
\vec{XZ} = (7-1, -7-(-2)) = (6, -5)
\]

\[
\vec{v} \cdot \vec{XZ} = (-2) \cdot 6 + 10 \cdot (-5) = -12 - 50 = -62
\]

\section*{Задание 8e}
Выяснить, являются ли данные векторы перпендикулярными:

\[
\vec{m} = \left( -\frac{3}{4}; -1 \right), \quad \vec{n} = (4; 2)
\]

\[
\vec{m} \cdot \vec{n} = \left( -\frac{3}{4} \right) \cdot 4 + (-1) \cdot 2 = -3 - 2 = -5
\]

\[
\vec{m} \cdot \vec{n} \neq 0 \quad \Rightarrow \quad \vec{m} \text{ не перпендикулярен } \vec{n}
\]

\end{document}