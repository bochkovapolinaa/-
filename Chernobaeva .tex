\documentclass[12pt]{article}
\usepackage[T2A]{fontenc}
\usepackage[utf8]{inputenc}
\usepackage[russian]{babel}
\usepackage{amsmath}
\usepackage{amssymb}
\usepackage{geometry}
\geometry{a4paper, margin=1.5cm}

\setlength{\parindent}{0pt}
\setlength{\parskip}{1em}


\begin{document}


\section*{Задача}

Найти \(A \cdot B\) и \(B \cdot A\):
\[
A = \begin{pmatrix}
3 & 4 & 1 \\
0 & -2 & 3 \\
2 & 1 & -1
\end{pmatrix},
\qquad
B = \begin{pmatrix}
1 & -2 & 1 \\
0 & 3 & 8 \\
4 & 1 & -2
\end{pmatrix}
\]

\subsection*{ Решение}

\subsection*{ \(A \cdot B\)}

Каждый элемент матрицы-произведения получается как сумма произведений элементов строки первой матрицы на соответствующие элементы столбца второй матрицы.

\[
\begin{pmatrix}
3 & 4 & 1 \\
0 & -2 & 3 \\
2 & 1 & -1
\end{pmatrix}
\begin{pmatrix}
1 & -2 & 1 \\
0 & 3 & 8 \\
4 & 1 & -2
\end{pmatrix}
=
\]

\[
=
\begin{pmatrix}
3\cdot1 + 4\cdot0 + 1\cdot4 & 3\cdot(-2) + 4\cdot3 + 1\cdot1 & 3\cdot1 + 4\cdot8 + 1\cdot(-2) \\
0\cdot1 + (-2)\cdot0 + 3\cdot4 & 0\cdot(-2) + (-2)\cdot3 + 3\cdot1 & 0\cdot1 + (-2)\cdot8 + 3\cdot(-2) \\
2\cdot1 + 1\cdot0 + (-1)\cdot4 & 2\cdot(-2) + 1\cdot3 + (-1)\cdot1 & 2\cdot1 + 1\cdot8 + (-1)\cdot(-2)
\end{pmatrix}
=
\]

\[
=
\begin{pmatrix}
7 & 7 & 33 \\
12 & -3 & -22 \\
-2 & -2 & 12
\end{pmatrix}
\]


\subsection*{ \(B \cdot A\)}

Каждый элемент матрицы-произведения получается как сумма произведений элементов строки первой матрицы на соответствующие элементы столбца второй матрицы.

\[
\begin{pmatrix}
1 & -2 & 1 \\
0 & 3 & 8 \\
4 & 1 & -2
\end{pmatrix}
\begin{pmatrix}
3 & 4 & 1 \\
0 & -2 & 3 \\
2 & 1 & -1
\end{pmatrix}
=
\]

\[
=
\begin{pmatrix}
1\cdot3 + (-2)\cdot0 + 1\cdot2 & 1\cdot4 + (-2)\cdot(-2) + 1\cdot1 & 1\cdot1 + (-2)\cdot3 + 1\cdot(-1) \\
0\cdot3 + 3\cdot0 + 8\cdot2 & 0\cdot4 + 3\cdot(-2) + 8\cdot1 & 0\cdot1 + 3\cdot3 + 8\cdot(-1) \\
4\cdot3 + 1\cdot0 + (-2)\cdot2 & 4\cdot4 + 1\cdot(-2) + (-2)\cdot1 & 4\cdot1 + 1\cdot3 + (-2)\cdot(-1)
\end{pmatrix}
=
\]

\[
=
\begin{pmatrix}
5 & 9 & -6 \\
16 & 2 & 1 \\
8 & 12 & 9
\end{pmatrix}
\]
\subsection*{Ответ}
\[
A \cdot B = \begin{pmatrix}
7 & 7 & 33 \\
12 & -3 & -22 \\
-2 & -2 & 12
\end{pmatrix},
\qquad
B \cdot A = \begin{pmatrix}
5 & 9 & -6 \\
16 & 2 & 1 \\
8 & 12 & 9
\end{pmatrix}
\]

\section*{Задача }
Найти частное
\[
\frac{7 - 2i}{-3 + 2i}
\]

\subsection*{Решение}

Для деления комплексных чисел умножим числитель и знаменатель на сопряжённое знаменателя:

\[
\frac{7 - 2i}{-3 + 2i} = \frac{(7 - 2i)(-3 - 2i)}{(-3 + 2i)(-3 - 2i)} =  \frac{-25 - 8i}{13} = -\frac{25}{13} - \frac{8}{13}i
\]

\subsection*{Ответ}
\[
-\frac{25}{13} - \frac{8}{13}i
\]

\section*{Задача}
Возвести в степень
\[
(-3 + 3i)^4\
\]
\subsection*{Решение}

Найдём модуль комплексного числа:
\[
|z| = \sqrt{(-3)^2 + 3^2} = \sqrt{18} = 3\sqrt{2}
\]

Найдем аргумент с помощью косинуса и синуса:
\[
\cos\varphi = \frac{-3}{3\sqrt{2}} = -\frac{1}{\sqrt{2}}, \quad \sin\varphi = \frac{3}{3\sqrt{2}} = \frac{1}{\sqrt{2}}
\]
Так как точка находится во второй четверти, то \(\varphi = \frac{3\pi}{4}\).

Тригонометрическая форма:
\[
z = 3\sqrt{2}\left(\cos\frac{3\pi}{4} + i\sin\frac{3\pi}{4}\right)
\]
По формуле Муавра:
\[
z^4 = (3\sqrt{2})^4\left(\cos\left(4\cdot\frac{3\pi}{4}\right) + i\sin\left(4\cdot\frac{3\pi}{4}\right)\right)
\]\
\[
z^4 = 324(-1 + 0i) = -324
\]
\subsection*{Ответ }
\[
(-3 + 3i)^4 = -324
\]

\section*{Задача}
Найти угол между векторами \(g\) и \(h\)
\[
g = (9; -5), \qquad h = (-0.8; 2)
\]
\subsection*{Решение}
Скалярное произведение векторов:
\[
g \cdot h = 9 \cdot (-0.8) + (-5) \cdot 2 = -17.2
\]

Длина вектора \(g\):
\[
|g| = \sqrt{9^2 + (-5)^2} = \sqrt{106}
\]

Длина вектора \(h\):
\[
|h| = \sqrt{(-0.8)^2 + 2^2} = \sqrt{4.64} = \frac{2\sqrt{29}}{5}
\]

Косинус угла между векторами:
\[
\cos\theta = \frac{g \cdot h}{|g| \cdot |h|} = \frac{-17.2}{\sqrt{106} \cdot \frac{2\sqrt{29}}{5}} = \frac{-43}{\sqrt{3074}}
\]

\subsection*{Ответ}
\[
\theta = \arccos\left(\frac{-43}{\sqrt{3074}}\right)
\]

\end{document}