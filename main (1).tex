\documentclass{article}
\usepackage{graphicx} % Required for inserting images

\title{Коллоквиум1 пересдача}
\author{Алина Серебрякова}
\date{December 2025}
\usepackage{amsmath}
\usepackage[utf8]{inputenc}
\usepackage[russian]{babel}
\usepackage[a4paper, margin=2cm]{geometry}

\begin{document}

\noindent\textbf{b) Дано:} \(A = \begin{pmatrix} -4 & 5 \\ 2 & 7 \end{pmatrix}\), \quad \(B = \begin{pmatrix} -3 & -3 \\ 0 & 11 \end{pmatrix}\)\\
\textbf\\{Найти:} \(1)A+B, \; 2)A-B\)

\noindent\textbf{1) Сложение матриц осуществляется путем складывания элементов, стоящих на одинаковых позициях:}
\[ (A + B)_{ij} = A_{ij} + B_{ij} \]
\textit{Операция возможна только, если матрицы имеют одинаковый размер (одинаковое число строк и столбцов).}
\[A+B= \begin{pmatrix} -4 & 5 \\ 2 & 7 \end{pmatrix} + \begin{pmatrix} -3 & -3 \\ 0 & 11 \end{pmatrix} = \begin{pmatrix} -7 & 2 \\ 2 & 18 \end{pmatrix}\]

\[ a_{11} = -4 + (-3) = -7 \]  
\[ a_{12} = 5 + (-3) = 2 \]  
\[ a_{21} = 2 + 0 = 2 \]  
\[ a_{22} = 7 + 11 = 18 \]  

\vspace{0.5cm}
\noindent
\textbf{2) Вычитание матриц осуществляется путем вычитания элементов, стоящих на одинаковых позициях:}
\[ (A-B)_{ij}=A_{ij}-B_{ij} \]
\textit{Операция возможна только, если матрицы имеют одинаковый размер (одинаковое число строк и столбцов).}
\[A-B = \begin{pmatrix} -4 & 5 \\ 2 & 7 \end{pmatrix} - \begin{pmatrix} -3 & -3 \\ 0 & 11 \end{pmatrix} = \begin{pmatrix} -1 & 8 \\ 2 & -4 \end{pmatrix} \]

\[ a_{11} = -4 - (-3) = -1 \]
\[ a_{12} = 5 - (-3) = 8 \]
\[ a_{21} = 2 - 0 = 2 \]
\[ a_{22} = 7 - 11 = -4 \]

\vspace{1cm}
\noindent\textbf{h) Дано:} \(A = \begin{pmatrix} -4 & 71 \\ 22 & 9 \\ 17 & -19 \end{pmatrix}\), \quad \(B = \begin{pmatrix} 52 & 12 \\ -5 & 0 \\ 100 & -24 \end{pmatrix}\)\\
\textbf\\{Найти:} \(1)A+B, \; 2)A-B\)

\noindent
1)  
\[ A + B = \begin{pmatrix} -4 & 71 \\ 22 & 9 \\ 17 & -19 \end{pmatrix} + \begin{pmatrix} 52 & 12 \\ -5 & 0 \\ 100 & -24 \end{pmatrix} = \begin{pmatrix} 48 & 83 \\ 17 & 9 \\ 117 & -43 \end{pmatrix} \]\\

\[ a_{11} = -4 + 52 = 48 \]
\[ a_{12} = 71 + 12 = 83 \]
\[ a_{21} = 22 + (-5) = 17 \]
\[ a_{22} = 9 + 0 = 9 \]
\[ a_{31} = 17 + 100 = 117 \]
\[ a_{32} = -19 + (-24) = -43 \]

\vspace{0.5cm}
\noindent
2)  
\[ A - B = \begin{pmatrix} -4 & 71 \\ 22 & 9 \\ 17 & -19 \end{pmatrix} - \begin{pmatrix} 52 & 12 \\ -5 & 0 \\ 100 & -24 \end{pmatrix} = \begin{pmatrix} -56 & 59 \\ 27 & 9 \\ -83 & 5 \end{pmatrix} \]\\

\[ a_{11} = -4 - 52 = -56 \]
\[ a_{12} = 71 - 12 = 59 \]
\[ a_{21} = 22 - (-5) = 27 \]
\[ a_{22} = 9 - 0 = 9 \]
\[ a_{31} = 17 - 100 = -83 \]
\[ a_{32} = -19 - (-24) = 5 \]

\end{document}