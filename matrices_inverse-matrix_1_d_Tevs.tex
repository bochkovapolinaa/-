

\documentclass[12pt]{article}
\usepackage[T2A]{fontenc}
\usepackage[utf8]{inputenc}
\usepackage[russian]{babel}
\usepackage{amsmath}
\usepackage{amssymb}
\usepackage{geometry}
\geometry{a4paper, margin=1.5cm}

\setlength{\parindent}{0pt}
\setlength{\parskip}{1em}

\begin{document}

\section*{Задача}
Вычислить:
\[
\begin{pmatrix}-3 & 7 & 6 \\-5 & 0 & -2 \end{pmatrix} \pm \begin{pmatrix}12 & 4 & 3\\-2 & -3 & 1 \end{pmatrix}
\]


\subsection*{Решение}
Поскольку в выражении указан знак \(\pm\), вычислим оба варианта.

Над данными матрицами можно проводить действия сложения и вычитания, т.к они одинакового размера (у них равное количество строк и столбцов).
Складываем по принципу: a11 одной матрицы с a11 другой матрицы. Вычитание проводится аналогично.

1. \textbf{Сложение}:
\[
\begin{pmatrix}-3 & 7 & 6 \\-5 & 0 & -2 \end{pmatrix} + \begin{pmatrix}12 & 4 & 3\\-2 & -3 & 1 \end{pmatrix} = 
\begin{pmatrix} -3+12 & 7+4 & 6+3 \\ -5+(-2) & 0+(-3) & -2+1 \end{pmatrix} = 
\begin{pmatrix} 9 & 11 & 9 \\ -7 & -3 & -1 \end{pmatrix}
\]

2. \textbf{Вычитание}:
\[
\begin{pmatrix}-3 & 7 & 6 \\-5 & 0 & -2 \end{pmatrix} - \begin{pmatrix}12 & 4 & 3\\-2 & -3 & 1 \end{pmatrix} = 
\begin{pmatrix} -3-12 & 7-4 & 6-3 \\ -5-(-2) & 0-(-3) & -2-1 \end{pmatrix} = 
\begin{pmatrix} -15 & 3 & 3 \\ -3 & 3 & -3 \end{pmatrix}
\]

\subsection*{Ответ:}
\[
\begin{pmatrix}-3 & 7 & 6 \\-5 & 0 & -2 \end{pmatrix} + \begin{pmatrix}12 & 4 & 3\\-2 & -3 & 1 \end{pmatrix} = \begin{pmatrix} 9 & 11 & 9 \\ -7 & -3 & -1 \end{pmatrix}
\]
\[
\begin{pmatrix}-3 & 7 & 6 \\-5 & 0 & -2 \end{pmatrix} - \begin{pmatrix}12 & 4 & 3\\-2 & -3 & 1 \end{pmatrix} = \begin{pmatrix} -15 & 3 & 3 \\ -3 & 3 & -3 \end{pmatrix}
\]

\end{document}