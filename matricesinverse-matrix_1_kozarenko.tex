\documentclass[12pt]{article}
\usepackage[T2A]{fontenc}
\usepackage[utf8]{inputenc}
\usepackage[russian]{babel}
\usepackage{amsmath}
\usepackage{amssymb}
\usepackage{geometry}
\geometry{a4paper, margin=1.5cm}

\setlength{\parindent}{0pt}
\setlength{\parskip}{1em}

\begin{document}

% ========== НАЧАЛО ЗАДАЧИ ==========
\section*{Задача}
Найти обратную матрицу \( A^{-1} \) к матрице \( A \):

а) 
\[
A = \begin{pmatrix}
1 & 1 \\
1 & -1
\end{pmatrix}
\]

\subsection*{Решение}
Для матрицы \( A = \begin{pmatrix} a & b \\ c & d \end{pmatrix} \) обратная матрица вычисляется по формуле:
\[
A^{-1} = \frac{1}{ad - bc} \begin{pmatrix} d & -b \\ -c & a \end{pmatrix},
\]
при условии, что определитель \( \det A = ad - bc \neq 0 \).

В нашем случае \( a = 1 \), \( b = 1 \), \( c = 1 \), \( d = -1 \).

1. Вычислим определитель:
\[
\det A = a d - b c = 1 \cdot (-1) - 1 \cdot 1 = -1 - 1 = -2.
\]

2. Так как определитель не равен нулю, обратная матрица существует. Подставим значения в формулу:
\[
A^{-1} = \frac{1}{-2} \begin{pmatrix} -1 & -1 \\ -1 & 1 \end{pmatrix}
= \begin{pmatrix} \frac{-1}{-2} & \frac{-1}{-2} \\ \frac{-1}{-2} & \frac{1}{-2} \end{pmatrix}
= \begin{pmatrix} \frac{1}{2} & \frac{1}{2} \\ \frac{1}{2} & -\frac{1}{2} \end{pmatrix}.
\]

\subsection*{Проверка} (для уверенности):
\[
A \cdot A^{-1} = 
\begin{pmatrix} 1 & 1 \\ 1 & -1 \end{pmatrix} \cdot 
\begin{pmatrix} \frac{1}{2} & \frac{1}{2} \\ \frac{1}{2} & -\frac{1}{2} \end{pmatrix} = 
\begin{pmatrix} 
1\cdot\frac{1}{2}+1\cdot\frac{1}{2} & 1\cdot\frac{1}{2}+1\cdot\left(-\frac{1}{2}\right) \\ 
1\cdot\frac{1}{2}+(-1)\cdot\frac{1}{2} & 1\cdot\frac{1}{2}+(-1)\cdot\left(-\frac{1}{2}\right) 
\end{pmatrix} = 
\begin{pmatrix} 1 & 0 \\ 0 & 1 \end{pmatrix}.
\]

\subsection*{Ответ:}
\[
A^{-1} = \begin{pmatrix} \frac{1}{2} & \frac{1}{2} \\ \frac{1}{2} & -\frac{1}{2} \end{pmatrix}.
\]

\end{document}