\documentclass[12pt]{article}
\usepackage[T2A]{fontenc}
\usepackage[utf8]{inputenc}
\usepackage[russian]{babel}
\usepackage{amsmath}
\usepackage{amssymb}
\usepackage{geometry}
\geometry{a4paper, margin=1.5cm}

\setlength{\parindent}{0pt}
\setlength{\parskip}{1em}

\begin{document}

\section*{Задача}
Дана система:
\[
\begin{cases}
x + 2y - z = 1,\\
2x + 3y + z = 2,\\
x + 3y - 2z = 1.
\end{cases}
\]

Чтобы решить эту систему с помощью матриц, представим её в матричной форме: 
\[
A \cdot X = B,
\] 
где:

\[
A = \begin{pmatrix}
1 & 2 & -1 \\
2 & 3 & 1 \\
1 & 3 & -2
\end{pmatrix}
\quad\text{--- матрица коэффициентов,}
\]

\[
X = \begin{pmatrix}
x \\ y \\ z
\end{pmatrix}
\quad\text{--- матрица неизвестных (вектор),}
\]

\[
B = \begin{pmatrix}
1 \\ 2 \\ 1
\end{pmatrix}
\quad\text{--- матрица правых частей (вектор свободных коэффициентов).}
\]

\subsection*{Шаг 1. Вычисление определителя}
Вычислив \(\det(A)\) мы можем узнать, существует решение или нет и единственное ли оно:
\begin{itemize}
    \item если \(\det(A) \neq 0\) \(\Rightarrow\) единственное решение
    \item если \(\det(A) = 0\) \(\Rightarrow\) либо решений нет, либо их бесконечное множество, что в некотором смысле одно и то же.
\end{itemize}

Вычислим \(\det(A)\) методом треугольников (правило Саррюса):
\begin{align*}
\det(A) &= 1 \cdot 3 \cdot (-2) + 2 \cdot 1 \cdot 1 + 2 \cdot 3 \cdot (-1) \\
       &\quad - (-1) \cdot 3 \cdot 1 - 1 \cdot 3 \cdot 1 - 2 \cdot 2 \cdot (-2) \\
       &= -6 + 2 - 6 + 3 - 3 + 8 = -2 \\
\end{align*}

\(\det(A) = -2 \neq 0\) \(\Rightarrow\) система имеет единственное решение.

\subsection*{Шаг 2. Нахождение решения через обратную матрицу}
Чтобы найти \(X\), нужно оставить его одного с одной из сторон уравнения. Сделать это можно с помощью обратной матрицы:
\[
A \cdot X = B \quad \Leftrightarrow \quad A^{-1} \cdot A \cdot X = A^{-1} \cdot B \quad \Leftrightarrow \quad X = A^{-1} \cdot B.
\]

Найдем обратную матрицу \(A^{-1}\). Формула для обратной матрицы:
\[
A^{-1} = \frac{1}{\det(A)} \cdot \operatorname{adj}(A),
\]
где \(\operatorname{adj}(A)\) --- это присоединённая матрица, которая является транспонированной матрицей алгебраических дополнений.

Алгебраическое дополнение элемента \(a_{ij}\) - это \(A_{ij}\), такая, что \(A_{ij} = (-1)^{i+j} \cdot M_{ij}\), где \(M_{ij}\) - минор элемента \(a_{ij}\).
Минор - это определитель матрицы, которая получается путём удаления \(i\)-той строки и \(j\)-того столбца.
Сначала найдём матрицу \(C\), состоящую из алгебраических дополнений элементов матрицы \(A\).

\subsubsection*{2.1. Находим матрицу алгебраических дополнений \(C\):}
\begin{align*}
C_{11} &= (-1)^{1+1} \cdot M_{11} = + \begin{vmatrix} 3 & 1 \\ 3 & -2 \end{vmatrix} = 3 \cdot (-2) - 1 \cdot 3 = -6 - 3 = -9, \\
C_{12} &= (-1)^{1+2} \cdot M_{12} = - \begin{vmatrix} 2 & 1 \\ 1 & -2 \end{vmatrix} = -[2 \cdot (-2) - 1 \cdot 1] = -[-4 - 1] = 5, \\
C_{13} &= (-1)^{1+3} \cdot M_{13} = + \begin{vmatrix} 2 & 3 \\ 1 & 3 \end{vmatrix} = 2 \cdot 3 - 3 \cdot 1 = 6 - 3 = 3, \\
C_{21} &= (-1)^{2+1} \cdot M_{21} = - \begin{vmatrix} 2 & -1 \\ 3 & -2 \end{vmatrix} = -[2 \cdot (-2) - (-1) \cdot 3] = -[-4 + 3] = 1, \\
C_{22} &= (-1)^{2+2} \cdot M_{22} = + \begin{vmatrix} 1 & -1 \\ 1 & -2 \end{vmatrix} = 1 \cdot (-2) - (-1) \cdot 1 = -2 + 1 = -1, \\
C_{23} &= (-1)^{2+3} \cdot M_{23} = - \begin{vmatrix} 1 & 2 \\ 1 & 3 \end{vmatrix} = -[1 \cdot 3 - 2 \cdot 1] = -[3 - 2] = -1, \\
C_{31} &= (-1)^{3+1} \cdot M_{31} = + \begin{vmatrix} 2 & -1 \\ 3 & 1 \end{vmatrix} = 2 \cdot 1 - (-1) \cdot 3 = 2 + 3 = 5, \\
C_{32} &= (-1)^{3+2} \cdot M_{32} = - \begin{vmatrix} 1 & -1 \\ 2 & 1 \end{vmatrix} = -[1 \cdot 1 - (-1) \cdot 2] = -[1 + 2] = -3, \\
C_{33} &= (-1)^{3+3} \cdot M_{33} = + \begin{vmatrix} 1 & 2 \\ 2 & 3 \end{vmatrix} = 1 \cdot 3 - 2 \cdot 2 = 3 - 4 = -1.
\end{align*}

Получаем матрицу алгебраических дополнений:
\[
C = \begin{pmatrix}
-9 & 5 & 3 \\
1 & -1 & -1 \\
5 & -3 & -1
\end{pmatrix}.
\]

Чтобы транспонировать матрицу, нужно поменять строки и столбцы местами.

\subsubsection*{2.2. Находим присоединённую матрицу \(\operatorname{adj}(A)\):}
Присоединённая матрица --- это транспонированная матрица алгебраических дополнений:
\[
\operatorname{adj}(A) = C^T = \begin{pmatrix}
-9 & 1 & 5 \\
5 & -1 & -3 \\
3 & -1 & -1
\end{pmatrix}.
\]

\subsubsection*{2.3. Находим обратную матрицу \(A^{-1}\):}
\[
A^{-1} = \frac{1}{\det(A)} \cdot \operatorname{adj}(A) = \frac{1}{-2} \cdot \begin{pmatrix}
-9 & 1 & 5 \\
5 & -1 & -3 \\
3 & -1 & -1
\end{pmatrix} = \begin{pmatrix}
\frac{9}{2} & -\frac{1}{2} & -\frac{5}{2} \\[2pt]
-\frac{5}{2} & \frac{1}{2} & \frac{3}{2} \\[2pt]
-\frac{3}{2} & \frac{1}{2} & \frac{1}{2}
\end{pmatrix}.
\]

\subsubsection*{2.4. Находим решение системы \(X = A^{-1} \cdot B\):}
\[
X = A^{-1} \cdot B = \begin{pmatrix}
\frac{9}{2} & -\frac{1}{2} & -\frac{5}{2} \\[2pt]
-\frac{5}{2} & \frac{1}{2} & \frac{3}{2} \\[2pt]
-\frac{3}{2} & \frac{1}{2} & \frac{1}{2}
\end{pmatrix} \cdot \begin{pmatrix}
1 \\ 2 \\ 1
\end{pmatrix}.
\]

Выполняем умножение:
\begin{align*}
x &= \frac{9}{2} \cdot 1 + \left(-\frac{1}{2}\right) \cdot 2 + \left(-\frac{5}{2}\right) \cdot 1 = \frac{9}{2} - 1 - \frac{5}{2} = \frac{9}{2} - \frac{2}{2} - \frac{5}{2} = \frac{2}{2} = 1, \\
y &= \left(-\frac{5}{2}\right) \cdot 1 + \frac{1}{2} \cdot 2 + \frac{3}{2} \cdot 1 = -\frac{5}{2} + 1 + \frac{3}{2} = -\frac{5}{2} + \frac{2}{2} + \frac{3}{2} = 0, \\
z &= \left(-\frac{3}{2}\right) \cdot 1 + \frac{1}{2} \cdot 2 + \frac{1}{2} \cdot 1 = -\frac{3}{2} + 1 + \frac{1}{2} = -\frac{3}{2} + \frac{2}{2} + \frac{1}{2} = 0.
\end{align*}

\subsection*{Проверка}
Подставим найденные значения в исходные уравнения:
\begin{align*}
&1)\; x + 2y - z = 1 + 2 \cdot 0 - 0 = 1, \quad\text{(верно)} \\
&2)\; 2x + 3y + z = 2 \cdot 1 + 3 \cdot 0 + 0 = 2, \quad\text{(верно)} \\
&3)\; x + 3y - 2z = 1 + 3 \cdot 0 - 2 \cdot 0 = 1. \quad\text{(верно)}
\end{align*}

\subsection*{Ответ}
\[
\boxed{x = 1,\quad y = 0,\quad z = 0}.
\]

\end{document}